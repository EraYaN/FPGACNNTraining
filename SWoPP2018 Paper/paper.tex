\documentclass[techrep,english]{ipsj}
\usepackage[utf8]{inputenc}


\usepackage[dvips]{graphicx}
\usepackage{latexsym}
\usepackage{amsmath}
\usepackage{adjustbox}

\def\Underline{\setbox0\hbox\bgroup\let\\\endUnderline}
\def\endUnderline{\vphantom{y}\egroup\smash{\underline{\box0}}\\}
\def\|{\verb|}

\setcounter{volume}{26} % vol21=2013, 22=14, 23=15, 24=16, 25=17, 26=18
\setcounter{number}{1}
\setcounter{page}{1}

%\received{2011}{7}{1}
%\rereceived{2011}{10}{1}   % optiona
%\rerereceived{2011}{10}{31} % optional
%\accepted{2011}{11}{5}

\usepackage[varg]{txfonts}%%!!
\makeatletter%
\input{ot1txtt.fd}
\makeatother%

\begin{document}

\title{Accelerating Deep Neural Network Training on FPGAs Facilitating the Use of Variable Precision}

\affiliate{TiTech}{Tokyo Institute of Technology, 
Meguro, Tokyo 152--8550, Japan}
\affiliate{TUDelft}{Delft University of Technology, Mekelweg 2, 2628 CD Delft, Netherlands}

\author{Erwin de Haan}{TiTech,TUDelft}[e.r.dehaan@student.tudelft.nl]
\author{Arthur Podobas}{TiTech}[podobas.a.aa@m.titech.ac.jp]
\author{Satoshi Matsuoka}{TiTech}

\begin{abstract}
The race for larger and deeper neural networks are leading researchers, vendors and practitioners to re-think architectural design decisions taken decades ago in hope to improve performance.
Among these decisions, reducing the numerical format is thought to be one of prime candidates to increasing performance.
Unfortunately, modern hardware has limited support for this, and the impact of modifying floating-point formats and its size remains shrouded in mystery.
To help investigate what the effects of varying the precision during deep-learning training are, first an architecture using reconfigurable hardware needs to be developed.
Through this work, we seek to accelerate arbitrary precision deep-learning training using Field-Programmable Gate-Arrays by leveraging Intel FPGA SDK for OpenCL.
Preliminary results show promise for implementations based on precisions or number formats that currently only have low performing implementations on conventional CPU and GPU hardware.
\end{abstract}

%\begin{keyword}
%Journal of Information Processing, \LaTeX, style files, ``Dos and
% Don'ts'' list
%\end{keyword}

\maketitle

%1
\section{Introduction}
Artur helps writing this.

\section{Related Work}
Artur can write this.

\section{A FPGA-based CNN Framework}
\subsection{Field-Programmable Gate-Arrays}
Small intro to FPGAs.

\subsection{(Convolutional) Neural Networks}
Small intro to Neural Networks.

\subsection{FPGA CNN Framework}
Small intro to the framework.
\subsubsection{Design overview}
Overview design + picture.
\subsubsection{Design details}
Gruesome details of the design.
\subsubsection{Design decisions}
Decisions taken while designing. The *why* we did as we did.

\section{Methodology}
\subsection{Experimental Platform}
Compiler, version, flags, fpga boards, host, etc.
\subsection{Network Tested + Data-set}
MNIST, Multilayer Perceptron?
number of runs, batch size, etc.

\section{Results}

\subsection{FPGA Resource Utilization}
Area, DSP, Fmax, etc.

\subsection{Training Performance}
Convergence, performance, accuracy, etc.

\begin{biography}
\profile{Joho Taro}{was born in 1970. He received his M.S.\ degree from
 Johoshori University in 1994 and has been engaged in the Information
 Processing Society of Japan since 1994. His research interest is online
 publishing systems. He is a member of the IEEE and ACM\@.}
%
\profile{Shori Hanako}{was born in 1960. She received her M.E.\ and
 Ph.D.\ from Johoshori University in 1984 and 1987, respectively. She
 became an associate professor at Gakkai University in 1992 and a
 professor at Johoshori University in 1997. Her current research
 interest is online publishing systems. She received the Kiyasu Kinen
 award in 2010. She is a Board Member of the IPSJ and a member of the
 IEICE, IEEE-CS, and ACM\@.} 
%
\profile{Gakkai Jiro}{was born in 1970. He received his M.S.\ degree
 from Johoshori University in 1994 and has been engaged in the
 Information Processing Society of Japan since 1994. His research
 interest is online publishing systems. He is a member of the IEEE and
 ACM\@.}
%
\end{biography}
\end{document}
